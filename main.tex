\documentclass[
	article,			
	11pt,				
	oneside,			
	a4paper,			
	english,			
	brazil,				
	sumario=tradicional
	]{abntex2}

\usepackage{lmodern}			
\usepackage[T1]{fontenc}		
\usepackage[utf8]{inputenc}		
\usepackage{indentfirst}		
\usepackage{nomencl} 			
\usepackage{color}				
\usepackage{graphicx}			
\usepackage{microtype}
\usepackage{lipsum}
\usepackage[brazilian,hyperpageref]{backref}
\usepackage[alf]{abntex2cite}
\usepackage{float}

\renewcommand{\backrefpagesname}{Citado na(s) página(s):~}

\renewcommand{\backref}{}

\renewcommand*{\backrefalt}[4]{
	\ifcase #1 %
		Nenhuma citação no texto.%
	\or
		Citado na página #2.%
	\else
		Citado #1 vezes nas páginas #2.%
	\fi}%


\titulo{Roubo de celulares no centro da cidade de São Paulo}
\tituloestrangeiro{}

\autor{
    \Large Centro Universitário Senac \\
    Bacharelado em Ciência da Computação 
\\[0.75cm]
\normalsize João Watson, Lucas Henrique, Murilo Cantante \\
\normalsize Priscila Wolff, Ricardo Hemmel, Thiago Pereira
}

\local{São Paulo}
\data{São Paulo, 30 de Maio 2022}

\definecolor{blue}{RGB}{41,5,195}

\makeatletter
\hypersetup{
        %pagebackref=true,
		pdftitle={\@title}, 
		pdfauthor={\@author},
        pdfsubject={Modelo de artigo científico com abnTeX2},
        pdfcreator={LaTeX with abnTeX2},
		pdfkeywords={abnt}{latex}{abntex}{abntex2}{atigo científico}, 
		colorlinks=true,       		% false: boxed links; true: colored links
        linkcolor=blue,          	% color of internal links
        citecolor=blue,        		% color of links to bibliography
        filecolor=magenta,      	% color of file links
		urlcolor=blue,
		bookmarksdepth=4
}
\makeatother

\makeindex

\setlrmarginsandblock{3cm}{3cm}{*}
\setulmarginsandblock{3cm}{3cm}{*}
\checkandfixthelayout

% O tamanho do parágrafo é dado por:
\setlength{\parindent}{1.3cm}

\setlength{\parskip}{0.2cm}

\SingleSpacing

\begin{document}

\selectlanguage{brazil}
\frenchspacing 
\maketitle

\begin{resumoumacoluna}
    
    Trata-se de um artigo com base no pensamento computacional, no qual possui o objetivo de 
    informar sobre como ocorre o roubo de celulares no centro da cidade de São Paulo e como
    eles nutrem um mercado ilegal, além de causar uma alta insegurança e um alto prejuízo ao
    cidadão brasileiro. Alguns casos ocorrem devido ao vício em drogas, fazendo com que o
    infrator roube celulares para comprar mais drogas. A solução encontrada para auxiliar na
    amenização desses casos é executar uma operação civil na região.
    
    \vspace{\onelineskip}
    
    \noindent
    \textbf{Palavras-chave}: roubo de celulares. vício em drogas. operação civil.
\end{resumoumacoluna}

\textual

\newpage
\section{Introdução}

    Apresentação detalhada da proposta do trabalho (qual o problema foi estudado,
    qual o contexto/a história do problema) e de sua justificativa (qual a
    contribuição do trabalho para a sociedade).

\newpage
\section{Objetivo}

    O objetivo do trabalho deve ser especificado detalhadamente.

\newpage
\section{Metodologia}

    Deve ser explicada detalhadamente

\newpage
\section{Discussão}

    Aqui deve ser discutido o problema tratado pelo trabalho, de modo fundamentado,
    com citações diretas e indiretas. Essa discussão deve ser feita de modo que
    contenha informações suficientes para o desenvolvimento do objetivo de trabalho.

\newpage
\section{Considerações finais}

    Fazer uma síntese do que foi discutido no trabalho, recuperando para o leitor qual
    foi o problema, estudando qual foi o objetivo deixando claro/evidente que o objetivo
    foi atingido.

\postextual

\newpage
\bibliography{references}

\end{document}
