\documentclass[
	article,			
	11pt,				
	oneside,			
	a4paper,			
	english,			
	brazil,				
	sumario=tradicional
	]{abntex2}

\usepackage{lmodern}			
\usepackage[T1]{fontenc}		
\usepackage[utf8]{inputenc}		
\usepackage{indentfirst}		
\usepackage{nomencl} 			
\usepackage{color}				
\usepackage{graphicx}			
\usepackage{microtype}
\usepackage{lipsum}
\usepackage[brazilian,hyperpageref]{backref}
\usepackage[alf]{abntex2cite}

\renewcommand{\backrefpagesname}{Citado na(s) página(s):~}

\renewcommand{\backref}{}

\renewcommand*{\backrefalt}[4]{
	\ifcase #1 %
		Nenhuma citação no texto.%
	\or
		Citado na página #2.%
	\else
		Citado #1 vezes nas páginas #2.%
	\fi}%


\titulo{Roubo de celulares no centro da cidade de São Paulo}
\tituloestrangeiro{}

\autor{
    \Large Centro Universitário Senac \\
    Bacharelado em Ciência da Computação 
\\[0.75cm]
\normalsize João Watson, Lucas Henrique, Murilo Cantante \\
\normalsize Priscila Wolff, Ricardo Hemmel, Thiago Pereira
}

\local{São Paulo}
\data{São Paulo, 30 de Maio 2022}

\definecolor{blue}{RGB}{41,5,195}

\makeatletter
\hypersetup{
        %pagebackref=true,
		pdftitle={\@title}, 
		pdfauthor={\@author},
        pdfsubject={Modelo de artigo científico com abnTeX2},
        pdfcreator={LaTeX with abnTeX2},
		pdfkeywords={abnt}{latex}{abntex}{abntex2}{atigo científico}, 
		colorlinks=true,       		% false: boxed links; true: colored links
        linkcolor=blue,          	% color of internal links
        citecolor=blue,        		% color of links to bibliography
        filecolor=magenta,      	% color of file links
		urlcolor=blue,
		bookmarksdepth=4
}
\makeatother

\makeindex

\setlrmarginsandblock{3cm}{3cm}{*}
\setulmarginsandblock{3cm}{3cm}{*}
\checkandfixthelayout

% O tamanho do parágrafo é dado por:
\setlength{\parindent}{1.3cm}

\setlength{\parskip}{0.2cm}

\SingleSpacing

\begin{document}

\selectlanguage{brazil}
\frenchspacing 
\maketitle

\begin{resumoumacoluna}
    % Conforme a ABNT NBR 6022:2018, o resumo no idioma do documento é elemento obrigatório. 
    % Constituído de uma sequência de frases concisas e objetivas e não de uma 
    % simples enumeração de tópicos, não ultrapassando 250 palavras, seguido, logo 
    % abaixo, das palavras representativas do conteúdo do trabalho, isto é,
    % palavras-chave e/ou descritores, conforme a NBR 6028. (\ldots) As 
    % palavras-chave devem figurar logo abaixo do resumo, antecedidas da expressão 
    % Palavras-chave:, separadas entre si por ponto e finalizadas também por ponto.
    
    Trata-se de um artigo com base no pensamento computacional, no qual possui o objetivo de 
    informar sobre como ocorre o roubo de celulares no centro da cidade de São Paulo e como
    eles nutrem um mercado ilegal, além de causar uma alta insegurança e um alto prejuízo ao
    cidadão brasileiro. Alguns casos ocorrem devido ao vício em drogas, fazendo com que o
    infrator roube celulares para comprar mais drogas. A solução encontrada para auxiliar na
    amenização desses casos é executar uma operação civil na região.
    
    \vspace{\onelineskip}
    
    \noindent
    \textbf{Palavras-chave}: roubo de celulares. vício em drogas. operação civil.
\end{resumoumacoluna}

\textual

\newpage
\section{Introdução}

Este documento e seu código-fonte são exemplos de referência de uso da classe
\textsf{abntex2} e do pacote \textsf{abntex2cite}. O documento exemplifica a
elaboração de publicação periódica científica impressa produzida conforme a ABNT
NBR 6022:2018 \emph{Informação e documentação - Artigo em publicação periódica
científica - Apresentação}.

A expressão ``Modelo canônico'' é utilizada para indicar que \abnTeX\ não é
modelo específico de nenhuma universidade ou instituição, mas que implementa tão
somente os requisitos das normas da ABNT. Uma lista completa das normas
observadas pelo \abnTeX\ é apresentada em \citeonline{abntex2classe}.

Sinta-se convidado a participar do projeto \abnTeX! Acesse o site do projeto em
\url{http://www.abntex.net.br/}. Também fique livre para conhecer,
estudar, alterar e redistribuir o trabalho do \abnTeX, desde que os arquivos
modificados tenham seus nomes alterados e que os créditos sejam dados aos
autores originais, nos termos da ``The \LaTeX\ Project Public
License''\footnote{\url{http://www.latex-project.org/lppl.txt}}.

Encorajamos que sejam realizadas customizações específicas deste documento.
Porém, recomendamos que ao invés de se alterar diretamente os arquivos do
\abnTeX, distribua-se arquivos com as respectivas customizações. Isso permite
que futuras versões do \abnTeX~não se tornem automaticamente incompatíveis com
as customizações promovidas. Consulte \citeonline{abntex2-wiki-como-customizar}
para mais informações.

Este exemplo deve ser utilizado como complemento do manual da classe
\textsf{abntex2} \cite{abntex2classe}, dos manuais do pacote
\textsf{abntex2cite} \cite{abntex2cite,abntex2cite-alf} e do manual da classe
\textsf{memoir} \cite{memoir}. Consulte o \citeonline{abntex2modelo} para obter
exemplos e informações adicionais de uso de \abnTeX\ e de \LaTeX.

\section{Exemplos de comandos}

\subsection{Margens}

A norma ABNT NBR 6022:2018 não estabelece uma margem específica a ser utilizada
no artigo científico. Dessa maneira, caso deseje alterar as margens, utilize os
comandos abaixo:

\begin{verbatim}
    \setlrmarginsandblock{3cm}{3cm}{*}
    \setulmarginsandblock{3cm}{3cm}{*}
    \checkandfixthelayout
\end{verbatim}

\subsection{Duas colunas}

É comum que artigos científicos sejam escritos em duas colunas. Para isso,
adicione a opção \texttt{twocolumn} à classe do documento, como no exemplo:

\begin{verbatim}
    \documentclass[article,11pt,oneside,a4paper,twocolumn]{abntex2}
\end{verbatim}

É possível indicar pontos do texto que se deseja manter em apenas uma coluna,
geralmente o título e os resumos. Os resumos em única coluna em documentos com
a opção \texttt{twocolumn} devem ser escritos no ambiente
\texttt{resumoumacoluna}:

\begin{verbatim}
    \twocolumn[              % INICIO DE ARTIGO EM DUAS COLUNAS

    \maketitle             % pagina de titulo

    \renewcommand{\resumoname}{Nome do resumo}
    \begin{resumoumacoluna}
        Texto do resumo.
        \vspace{\onelineskip}
        \noindent
        \textbf{Palavras-chave}: latex. abntex. editoração de texto.
    \end{resumoumacoluna}

    ]                        % FIM DE ARTIGO EM DUAS COLUNAS
\end{verbatim}

\subsection{Recuo do ambiente \texttt{citacao}}

Na produção de artigos (opção \texttt{article}), pode ser útil alterar o recuo
do ambiente \texttt{citacao}. Nesse caso, utilize o comando:

\begin{verbatim}
    \setlength{\ABNTEXcitacaorecuo}{1.8cm}
\end{verbatim}

Quando um documento é produzido com a opção \texttt{twocolumn}, a classe
\textsf{abntex2} automaticamente altera o recuo padrão de 4 cm, definido pela
ABNT NBR 10520:2002 seção 5.3, para 1.8 cm.

\section{Cabeçalhos e rodapés customizados}

Diferentes estilos de cabeçalhos e rodapés podem ser criados usando os
recursos padrões do \textsf{memoir}.

Um estilo próprio de cabeçalhos e rodapés pode ser diferente para páginas pares
e ímpares. Observe que a diferenciação entre páginas pares e ímpares só é
utilizada se a opção \texttt{twoside} da classe \textsf{abntex2} for utilizado.
Caso contrário, apenas o cabeçalho padrão da página par (\emph{even}) é usado.

Veja o exemplo abaixo cria um estilo chamado \texttt{meuestilo}. O código deve
ser inserido no preâmbulo do documento.

\begin{verbatim}
%%criar um novo estilo de cabeçalhos e rodapés
\makepagestyle{meuestilo}
    %%cabeçalhos
    \makeevenhead{meuestilo} %%pagina par
        {topo par à esquerda}
        {centro \thepage}
        {direita}
    \makeoddhead{meuestilo} %%pagina ímpar ou com oneside
        {topo ímpar/oneside à esquerda}
        {centro\thepage}
        {direita}
    \makeheadrule{meuestilo}{\textwidth}{\normalrulethickness} %linha
    %% rodapé
    \makeevenfoot{meuestilo}
        {rodapé par à esquerda} %%pagina par
        {centro \thepage}
        {direita} 
    \makeoddfoot{meuestilo} %%pagina ímpar ou com oneside
        {rodapé ímpar/onside à esquerda}
        {centro \thepage}
        {direita}
\end{verbatim}

Para usar o estilo criado, use o comando abaixo imediatamente após um dos
comandos de divisão do documento. Por exemplo:

\begin{verbatim}
    \begin{document}
        %%usar o estilo criado na primeira página do artigo:
        \pretextual
        \pagestyle{meuestilo}

        \maketitle
        ...

        %%usar o estilo criado nas páginas textuais
        \textual
        \pagestyle{meuestilo}

        \chapter{Novo capítulo}
        ...
    \end{document}  
\end{verbatim}

Outras informações sobre cabeçalhos e rodapés estão disponíveis na seção 7.3 do
manual do \textsf{memoir} \cite{memoir}.

\section{Mais exemplos no Modelo Canônico de Trabalhos Acadêmicos}

Este modelo de artigo é limitado em número de exemplos de comandos, pois são
apresentados exclusivamente comandos diretamente relacionados com a produção de
artigos.

Para exemplos adicionais de \abnTeX\ e \LaTeX, como inclusão de figuras,
fórmulas matemáticas, citações, e outros, consulte o documento
\citeonline{abntex2modelo}.

\section{Consulte o manual da classe \textsf{abntex2}}

Consulte o manual da classe \textsf{abntex2} \cite{abntex2classe} para uma
referência completa das macros e ambientes disponíveis.

\bookmarksetup{startatroot}% 

\section{Considerações finais}

\lipsum[1]

\begin{citacao}
\lipsum[2]
\end{citacao}

\lipsum[3]

\postextual

\bibliography{references}

\end{document}
