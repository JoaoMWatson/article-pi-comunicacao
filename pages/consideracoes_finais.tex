\section{Considerações finais}

    Ao decorrer desta pesquisa e considerando todos os pontos estudados e analisados,
    foi possível estabelecer uma relação entre o roubo de aparelhos celular e a
    crescente demanda de serviços de delivery. 

    Com o inicio pandemia e a crescente demanda de serviços de delivery, 
    fez com que a classe dos entregadores  crescesse como nunca antes visto, 
    esse fenômeno se deu pelo grande desemprego gerado pela Covid e a má 
    administração do governo vigente, juntamente com a facilidade de acesso aos 
    aparelhos celulares e internet domestica as pessoas passaram a pedir comida 
    e outros produtos com muito mais frequência, desde alimentos prontos, fast 
    food, mercado e até mesmo produtos farmacêuticos.

    Porem essa tendencia não diminuiu mesmo com o flexibilização da circulação 
    e a diminuição dos casos pois a população se acostumou com as facilidades 
    que esse tipo de serviço prove. Com as pessoas voltando as ruas,
    a grande quantidade de motoboys e a sucateamento da segurança publica, criou-se um cenário onde
    assaltantes disfarçados de entregadores cometem crimes contra o cidadão focando principalmente em 
    celulares sendo esse o item mais visado e rentável.

    Mostramos como fica cada vez mais vital a necessidade de regularização
    da profissão dos entregadores de aplicativo, para que, tanto o cidadão
    quanto os prestadores de serviço tenham mais segurança em seu dia a dia.

\postextual