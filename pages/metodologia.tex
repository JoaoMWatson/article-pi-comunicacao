\section{Metodologia}

Iniciaremos o presente artigo através do pensamento computacional, que envolve 
a aplicação de aspectos da computação em diversas áreas da vida, de força 
estratégica e crítica. O pensamento computacional não se diz sobre somente coisas
relacionadas a computador ou programação, se diz sobre lógica, objetividade e o 
uso da pesquisa para formular sugestões de melhora para um determinado assunto. 
Essa competência permite a construção de soluções para problemas comuns de forma inovadora.
Abstração e algoritmos são duas bases do pensamento computacional, e elas abrem espaço para 
uma tarefa importante que é a autonomia.