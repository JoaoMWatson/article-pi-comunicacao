\section{Metodologia}

Visto que nosso foco é propor uma alternativa ou proposta para esse problema
social que afeta vida de muitos moradores da região, foi determinado 
que seria utilizado o método do pensamento computacional para chegar até a 
tratativa do problema. Iniciamos o presente artigo através deste método, que 
envolve a aplicação de aspectos da computação em diversas áreas da vida, de força 
estratégica e crítica. O pensamento computacional não se diz sobre somente coisas 
relacionadas a computador ou programação, se diz sobre lógica, objetividade e o 
uso da pesquisa para formular soluções para um determinado assunto.

\subsection{Bases do pensamento computacional}

Abstração e algoritmos são duas bases do pensamento computacional, e elas 
abrem espaço para uma tarefa importante que é a autonomia. O pensamento 
computacional pode ser compreendido como uma forma de utilizar o 
computador não apenas como uma simples ferramenta para produzir textos ou 
acessar a internet, mas também como uma ferramenta poderosa que pode 
auxiliar a população na resolução de problemas. Além disso, o 
Pensamento Computacional transcende e amplia a definição de ser um 
conhecimento relacionado somente ao uso do computador, uma vez que pode 
ser utilizado nos mais diversos campos do conhecimento humano. Será utilizado
os seguintes passos:

\subsubsection{Decomposição}
Ao se deparar com um problema complexo, é possível dividi-lo em partes,
tornando-o em problemas mais simples.

\subsubsection{Reconhecimento de Padrões}
Nesta etapa é feito a identificação de padrões no processo para conseguir a
resolução do problema, observando o que se repete e como isso pode ser 
solucionado ou utilizado, dependendo da situação. 

\subsubsection{Abstração}
Durante esta etapa, é feita uma filtração dos padrões e situações que podem 
ser relevantes ou redundantes. 

\subsubsection{Algoritmos}
É a elaboração de um plano de ação no qual é feito um passo a passo para que 
determinado problema seja amenizado ou solucionado.