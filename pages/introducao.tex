\section{Introdução}

Com o início da pandemia, aplicativos e serviços de entrega começaram a se 
popularizar devido à quarentena que ocorreu no país. Nos noticiários, 
assaltantes disfarçados de entregadores de aplicativo começaram a 
aparecer cada vez mais. Neste artigo será analisado dados sobre roubos
de celulares registrados, com o foco na região do Grajaú.

Apesar de ser um delito bastante conhecido, o roubo de celulares possui em sua 
essência algumas características relevantes que merecem atenção, nos quais serão
vistos neste artigo. Um dos pontos que merecem a atenção é a diferença entre furto
e roubo, pois além de serem diferentes, possuem penalidades distintas. A diferença 
entre ambos pode ser vista no Título II, dos crimes contra o patrimônio, do Código Penal 
Brasileiro, o crime de roubo possui as mesmas características do crime de furto, 
porém, quando há o emprego de grave ameaça, de violência ou outro meio que 
impossibilite a resistência da vítima, fatores estes, empregados pelo agente para
que a vítima entregue o bem, está configurado o presente crime.