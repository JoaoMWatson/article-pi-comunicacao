\section{Objetivo}

    \subsection{Objetivos Gerais}

        % Este artigo visa estabelecer uma clara relação entre o roubo de aparelhos 
        % celular e os serviços de delivery, e como isso interfere na segurança pública
        % tendo como referência a região do Grajaú. Essa pesquisa se desenvolve analisando
        % esses casos e formas de contornar  essa situação no intuito de melhorar a 
        % situação dos moradores da região  escolhida. Desse modo, informar o leitor 
        % sobre o problema social da região do Grajaú  

        Este artigo visa analisar e informar sobre o roubo de aparelhos celulares envolvendo
        assaltantes disfarçados de entregadores de aplicativo, mostrando como isso interfere
        na segurança pública tendo como referência a região do Grajaú. Esse artigo desenvolve-se
        analisando casos e formas de contornar essa situação no intuito de melhorar a segurança
        dos moradores da região escolhida.

    \subsection{Objetivos Específicos}

    Nesse artigo discutimos a situação de segurança pública de um 
    distrito ao sul do município de São Paulo, mais especificamente
    a região do Grajaú. Desenvolvemos essa pesquisa olhando para os 
    casos de roubo de aparelho celular envolvendo entregadores de 
    aplicativo, assim relacionando dois tópicos muito recorrentes
    da nossa atualidade e mostrando os impactos que eles causam
    quando juntos. A popularização dos serviços de delivery durante
    a quarentena devido a pandemia da Covid-19 é um dos motivos pelo
    aumento exponencial dessa área de trabalho. Do outro lado também 
    temos o alto índice de roubos na região do Grajaú. Procuramos 
    apresentar para o leitor uma ideia do oportunismo diante essa 
    situação por parte dos assaltantes, tendo como um método efetivo
    de disfarce, uma profissão em ascenção no mercado de trabalho, que
    por sua vez já é exercída com a utilização de uma motocicleta,
    veículo esse muito ultilizado por bandidos, por facilitar o 
    acesso em locais mais fechados. Através de dados podemos observar
    a gravidade da situação, e quando estabelecemos
    estatíscas a partir desses dados é fácil de visualizar a gravidade 
    da situação dos furtos. Em busca de uma forma de reduzir os casos
    de roubos envolvendo o disfarce do bandido como trabalhador, foi 
    possível identificar algumas propostas já discutidas por autoridades
    e até mesmo por sindicatos dos motociclistas, como exigir mais 
    documentação dos trabalhadores dessa área ou uma 
    caracterização mais detalhada desse tipo de trabalhador. Nesse artigo 
    também é sugerido formas de evitar ou diminuir esses tipos de casos
    de acordo com um especialista, e com base nesses dados é dado uma sugestão 
    de melhora por parte dos pesquisadores desse artigo.