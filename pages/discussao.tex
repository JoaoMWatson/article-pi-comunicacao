\section{Discussão}

    % Aqui deve ser discutido o problema tratado pelo trabalho, de modo fundamentado,
    % com citações diretas e indiretas. Essa discussão deve ser feita de modo que
    % contenha informações suficientes para o desenvolvimento do objetivo de trabalho.
    Abaixo será discutido e analisado o que pode ser feito para diminuir a alta taxa de
    roubos no Grajaú, baseando-se em estatísticas e notícias deste e de outros bairros em 
    situações semelhantes da cidade de São Paulo.

    Esta seção foi separada em três tópicos: Origem, Dados e estatísticas e Sugestões de melhora.

    \subsection{Origem}

        Com a flexibilização da quarentena devido ao Covid-19 e a popularização de aplicativos
        e serviços de entrega de alimentos, os casos de assaltos começaram a aumentar,
        principalmente com assaltantes disfarçando-se de entregadores de aplicativos com o
        intuito de diminuir suspeitas e facilitar o assalto.
        
        Devido à sua valorização, uma das coisas que os assaltantes mais procuram roubar é o
        celular, pois além de seu preço, é fácil de ser transportado e quando a vítima está
        distraída com ele, torna-se ainda mais fácil de realizar o assalto.
        
        Outro fator que faz o celular ser prioridade aos assaltantes é a quantidade de informações
        que podem ter dentro dele, como dados bancários, senhas e contas.

    \subsection{Dados e estatísticas}

        Segundo a notícia feita pelo programa jornalístico SP1, no dia 19 de Maio de 2022,
        foram em todo o estado 26.484 furtos nos dois primeiros meses do ano e 34.338 roubos
        no mesmo período, de acordo com dados dos boletins de ocorrência. Este levantamento
        foi feito pela GloboNews. A notícia também apresenta que a cada hora 42 celulares
        são furtados ou roubados no estado de São Paulo. Grande parte das vítimas deste tipo
        de crime são pedestres no período da noite na cidade de São Paulo. Na capital,
        encontra-se o Grajaú como um dos bairros que possui os maiores números de ocorrências.

        O programa jornalístico também informa sobre um pacote de medidas preventivas
        do governo do estado de São Paulo, com o intuito de inibir a onda de ocorrências
        envolvendo assaltantes disfarçados de entregadores de aplicativos, realizando mais
        abordagens aos motociclistas e parcerias às empresas dos respectivos aplicativos para
        uma fiscalização mais efetiva.
        % https://g1.globo.com/sp/sao-paulo/noticia/2022/05/19/42-celulares-sao-roubados-ou-furtados-por-hora-no-estado-de-sp-em-janeiro-e-fevereiro.ghtml

    \subsection{Sugestões de melhora}

        De acordo com um especialista em segurança citado pelo programa jornalístico SP1, no dia 
        19 de Maio de 2022, há diversas coisas que podemos fazer para se proteger em caso de
        roubo de celular. "A primeira e mais simples é a senha. Não podem ser fáceis, dedutíveis
        ou estarem anotadas no celular. O que tem acontecido é roubo de celular quando ele está
        desbloqueado, seja por motoristas presos no celular ou pedestres utilizando-o na mão. O
        telefone na mão já é a primeira barreira derrubada, permitindo que os bandidos acessem
        nossos dados e até contas bancárias. Também todos nós devemos estar preparados para este 
        acidente. Quais são os telefones dos bancos que você tem de ligar? No momento de nervosismo,
        é importante ter esse roteiro na bolsa, no porta-luvas do carro, em casa, para você chegar
        e poder avisar que foi roubado.", o especialista afirma.
        % https://g1.globo.com/sp/sao-paulo/noticia/2022/05/19/42-celulares-sao-roubados-ou-furtados-por-hora-no-estado-de-sp-em-janeiro-e-fevereiro.ghtml

        Como sugestão de melhora, é possível realizar uma regularização para entregadores de
        aplicativo, evitando a possibilidade de assaltantes se disfarçarem e facilitando a
        fiscalização para as autoridades.